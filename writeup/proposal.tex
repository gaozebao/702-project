\iffalse
\documentclass[11pt, reqno]{amsart}
\usepackage{graphicx}
\usepackage{geometry}
\usepackage{hyperref}
\geometry{letterpaper}
\usepackage{float}
\usepackage{cite}

\begin{document}

\begin{center}
\textbf{Lagrangian relaxation for tree-structured covariance estimation}
\newline
\end{center}
\begin{flushright}
Xiyang Dai, Zebao Gao and Hao Zhou\\
Dept. of Computer Science
\end{flushright}
\fi

\section{Tree-structured Covariance Estimation}
Understanding similarities in expression profiles from a set of samples, e.g., multiple brain regions, is valuable to help understand many biological processes. A natural way to analyze this kind of similarities is to model a tree-structured covariance estimation. However, this kind of model turns out to be an non-convex optimization problems shown in recent paper from Hector etc \cite{Hector09}. In their paper, they have proposed a mixed-integer programming (MIP) approach to solve this non-convex optimization problem. 
Specifically, they estimate a covariance matrix from observations of p continuous random variables encoding a stochastic process over a tree with p leaves. To formulate the estimation problem as instances of weel-studied numerical optimization problems, they used linear combinations of rank-one matrices indicating object partitions. Although they have shown great performance using MIP, this non-convex optimization problem for tree-structure covariance estimation is still far from solved.

\section{Lagrangian Relaxation}
Recently, in Natural Language Processing (NLP) filed, Lagrangian relaxation methods are employed for solving non-convex optimization problems .

Recently, there is an increasing trend in applying Lagrangian relaxation methods to solve non-convex optimization problems, especially in NLP field. It has been successfully applied to several NLP inference problems such as Part-of-speech tagging \cite{Lagrangian}. But more studies are still desirable to conduct to evaluate its practicability in the tree-structured covariance estimation in our Bioinformatics problems.

\section{Our Project}
In this project, we will try to see whether Lagrangian relaxation method can be applied to solve this kind of non-convex optimization problem, and try to apply this tree-structured estimation method to Bio-related topics.
The main challenges of our project include:
\begin{itemize}
\item[1] understanding the existing tree-structured covariance estimation problem;
\item[2] understanding the Lagrangian relaxation method;
\item[3] try to apply the method from NLP field to bio-related topics;
\item[4] implementation of these methods.
\end{itemize}

\iffalse
\bibliography{mybib}
\bibliographystyle{plain}
\end{document}
\fi
